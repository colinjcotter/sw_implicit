\documentclass{article}
\usepackage{amsmath,amssymb,graphicx}
\def\MM#1{\boldsymbol{#1}}
\newcommand{\pp}[2]{\frac{\partial #1}{\partial #2}} 
\newcommand{\dede}[2]{\frac{\delta #1}{\delta #2}}
\newcommand{\dd}[2]{\frac{\diff#1}{\diff#2}}
\newcommand{\dt}[1]{\diff\!#1}
\def\MM#1{\boldsymbol{#1}}
\DeclareMathOperator{\diff}{d}
%\DeclareMathOperator{\Re}{Re}
%\DeclareMathOperator{\Im}{Im}
\DeclareMathOperator{\DIV}{DIV}
\DeclareMathOperator{\Hdiv}{H(div)}
\DeclareMathOperator{\D}{D}
\usepackage{natbib}
\bibliographystyle{elsarticle-harv}
\usepackage{helvet}
\usepackage{amsfonts}
\renewcommand{\familydefault}{\sfdefault} %% Only if the base font of the docume
\newcommand{\vecx}[1]{\MM{#1}}
\newtheorem{definition}{Definition}
\newcommand{\code}[1]{{\ttfamily #1}} 
\usepackage[margin=2cm]{geometry}

\usepackage{fancybox}
\begin{document}
\title{Notes on Augmented Lagrangian preconditioning for GFD}
\author{Tous Nous}
\maketitle

\section{Rotating shallow water}

Say that we have a continuous time mixed FEM formulation for the
rotating shallow water equations given by:
find $(u,D)\in V_1\times V_2$ such that
\begin{align}
  \langle w, u_t \rangle + A(w; u) - \langle \nabla\cdot w, gD\rangle
  & = 0, \quad \forall w \in V_1, \\
  \langle \phi, D_t \rangle + B(\phi; u, D) & = 0, \quad \forall \phi\in V_2.
\end{align}
Here $A$ is the nonlinear velocity advection term 1-form and $B$ is
the $D$ nonlinear advection 1-form.

We discretise in time using the implicit midpoint rule (doesn't really matter
which implicit scheme you use) to obtain: find $(u^{n+1},D^{n+1})
\in V_1\times V_2$ such that
\begin{align}
  R_u(w;u^{n+1},D^{n+1}) = \langle w, u^{n+1}-u^n \rangle + \Delta t A(w; u)
  + \Delta t\langle w, fu^{\perp}\rangle
  - \Delta t\langle \nabla\cdot w, gD^{n+1/2}\rangle
  & = 0, \quad \forall w \in V_1, \\
  R_D(\phi; u^{n+1},D^{n+1}) = \langle \phi, D^{n+1} - D^n \rangle
  + \Delta t B(\phi; u^{n+1/2},D^{n+1/2}) & = 0, \quad \forall \phi\in V_2,
\end{align}
where $u^{n+1/2} = (u^{n+1}+u^n)/2$, $D^{n+1/2}=(D^{n+1}+D^n)/2$.

We now modify this formulation in augmented Lagrangian style so that
it becomes: find $(u^{n+1},D^{n+1}) \in V_1\times V_2$ such that
\begin{align}
  \nonumber
  \langle w, u^{n+1}-u^n \rangle + \Delta t A(w; u^{n+1/2})
  + \Delta t\langle w, f(u^{n+1/2})^{\perp}\rangle
  - \Delta t\langle \nabla\cdot w, gD^{n+1/2}\rangle & \\
  \qquad + \gamma\left(
  \langle \nabla\cdot w, D^{n+1} - D^n \rangle
  + \Delta t B(\nabla\cdot w; u^{n+1/2},D^{n+1/2})
  \right)
  & = 0, \quad \forall w \in V_1, \\
  \langle \phi, D^{n+1} - D^n \rangle
  + \Delta t B(\phi; u^{n+1/2},D^{n+1/2}) & = 0, \quad \forall \phi\in V_2,
\end{align}
for some $\gamma>0$.  Here we have substituted $\phi\to \nabla\cdot w$
into the continuity equation before multiplying by $\gamma$ and adding
to the velocity equation. Since $\nabla\cdot w\in V_2$, any solution
of this augmented formulation is also a solution of the original
formulation.

We now consider Newton's method applied to the augmented system.

Writing $\beta=\Delta t/2$ and given iterative guesses for
$(u^{n+1},D^{n+1})$, the Newton updates $(u',D')$ satisfy
\begin{align}
  \nonumber
  \langle w, u' \rangle + \beta A'(w, u'; u^{n+1})
  + \beta\langle w, f(u')^{\perp}\rangle
  - \beta\langle \nabla\cdot w, gD'\rangle & \\
  \nonumber
  \qquad + \gamma\Big(
  \langle \nabla\cdot w, D' \rangle
  + \beta B_u(\nabla\cdot w, u'; u^{n+1},D^{n+1}) & \\
  \nonumber
  \qquad  + \beta B_D(\nabla\cdot w, D'; u^{n+1},D^{n+1})
  \Big)
  & = -R_u(w;u^{n+1},D^{n+1}) \\
  & \qquad \qquad - \gamma R_D(\nabla\cdot w;
  u^{n+1},D^{n+1}), \quad \forall w \in V_1, \\
  \langle \phi, D' \rangle
  + \beta B_u(\phi, u'; u^{n+1},D^{n+1})
  + \beta B_D(\phi, D'; u^{n+1},D^{n+1})
  & = -R_D(\phi; u^{n+1},D^{n+1}), \quad \forall \phi\in V_2.
\end{align}
Choosing $w \in \zeta_1$, where $\zeta_1=\{u\in V_1:\nabla\cdot u=0\}$,
we have
\begin{equation}
  \langle w, u'_\zeta\rangle + \beta A'(w,u'_\zeta + u'_{\zeta^\perp}; u^{n+1})
  + \beta\langle w, f(u'_\zeta + u'_{\zeta^\perp})^\perp\rangle = -R_u(w; u^{n+1},D^{n+1}),
  \quad \forall w\in \zeta_1,
\label{eq:uzeta}
\end{equation}
where $u'=u'_\zeta+u'_{\zeta^\perp}$ where $u'_\zeta\in \zeta_1$ and
$u'_{\zeta^{\perp}}\in \zeta_1^\perp$, the orthogonal complement of
$\zeta_1$ in $V_1$. If we choose instead $w\in \zeta_1^\perp$,
divide by $\gamma$ and send $\gamma\to \infty$, we get
\begin{align}
  \nonumber
   \beta B_u(\nabla\cdot w, u'_{\zeta^\perp}; u^{n+1},D^{n+1})  
  + \beta B_u(\nabla\cdot w, u'_\zeta; u^{n+1},D^{n+1}) & = 
  - \beta B_D(\nabla\cdot w, D'; u^{n+1},D^{n+1}) -\langle \nabla\cdot w, D' \rangle \\
  & \qquad -R_D(\nabla\cdot w;u^{n+1},D^{n+1}), \qquad \forall w \in \zeta_1^\perp,
\label{eq:uzetaperp}
\end{align}
where $u' = u'_{\zeta} + u'_{\zeta^\perp}$.  The solveability of this
equation for $u'_{\zeta^\perp}$ given $u'_{\zeta}$ and $D'$ needs a
bit of work, but $B_u$ is something related to $\nabla\cdot u$ so this
is essentially a grad-div operator that is solveable in
$\zeta^{\perp}_1$. Let's assume it for this note. Then, given $D'$, the
system (\ref{eq:uzeta}-\ref{eq:uzetaperp}) can be solved for both
components of $u'$, and we conclude that the dependence of $u'$ on
$D'$ is asymptotically independent of $\gamma$ as $\gamma\to \infty$.

For later, we'll define $\mathcal{L}_0:V_2\to V_1$ by splitting
$\mathcal{L}_0D'=v_\zeta + v_{\zeta^\perp}$, according to
\begin{align}
    \langle w, v_\zeta\rangle + \beta A'(w,v_\zeta + v_{\zeta^\perp}; u^{n+1})
  + \beta\langle w, f(v_\zeta + v_{\zeta^\perp})^\perp\rangle &= 0,
  \quad \forall w\in \zeta_1, \\
  \nonumber
   \beta B_u(\nabla\cdot w, v_{\zeta^\perp}; u^{n+1},D^{n+1})  
  + \beta B_u(\nabla\cdot w, v_\zeta; u^{n+1},D^{n+1}) & = 
  - \beta B_D(\nabla\cdot w, D'; u^{n+1},D^{n+1}) -\langle \nabla\cdot w, D' \rangle \\
  & \qquad \forall w \in \zeta_1^\perp.
\end{align}
Similarly, we'll define $r\in V_1$ with $r=r_\zeta + r_{\zeta^\perp}$ according to
\begin{align}
    \langle w, r_\zeta\rangle + \beta A'(w,r_\zeta + r_{\zeta^\perp}; u^{n+1})
  + \beta\langle w, f(r_\zeta + r_{\zeta^\perp})^\perp\rangle &= -R_u(w; u^{n+1},D^{n+1}),
  \quad \forall w\in \zeta_1, \\
   \beta B_u(\nabla\cdot w, r_{\zeta^\perp}; u^{n+1},D^{n+1})  
  + \beta B_u(\nabla\cdot w, r_\zeta; u^{n+1},D^{n+1}) & = 
-R_D(\nabla\cdot w;u^{n+1},D^{n+1}), \qquad \forall w \in \zeta_1^\perp,
\end{align}
so that $u_0 = \mathcal{L}_0D' + r$.

Now we use this to examine the asymptotics of the $\gamma\to \infty$
limit of the Schur complement upon eliminating $u'$. To do this, we
write the dependence of $u'$ on $D'$ as $u'(D') =u_0 + \gamma^{-1} u_1
+ \mathcal{O}(\gamma^{-2})$, which we just justified. At
$\mathcal{O}(1)$, we have
\begin{align}
  \nonumber \beta B_u(\nabla\cdot w, u_0; u^{n+1},D^{n+1}) & = - \beta
  B_D(\nabla\cdot w, D'; u^{n+1},D^{n+1}) \\ & \qquad -\langle
  \nabla\cdot w, D' \rangle -R_D(\nabla\cdot w;u^{n+1},D^{n+1}), \quad
  \forall w\in V_1.
\end{align}

Since for all $\phi\in V_2$ there exists $w\in V_1$ such that
$\nabla\cdot w = \phi$, the $\phi$ equation becomes $0=0$ at
$\mathcal{O}(1)$ (that's why we need to go one term further in the
expansion).

At the next order, we get
\begin{align}
  \nonumber
  \beta B_u(\nabla\cdot w, u_1; u^{n+1},D^{n+1})
  & = 
  -\langle w, u_0 \rangle - \beta A'(w, u_0; u^{n+1})
  - \beta\langle w, f(u_0)^{\perp}\rangle \\
  & \qquad\qquad
  + \beta\langle \nabla\cdot w, gD'\rangle
  -R_u(w;u^{n+1},D^{n+1})
  \quad \forall w \in V_1.
\end{align}

Defining $w_\phi$ as the unique function in $\zeta_1^{\perp}$ such
that $\nabla\cdot w_\phi=\phi$, we get
\begin{align}
  \nonumber
  \beta B_u(\phi, u_1; u^{n+1},D^{n+1})
  & = 
  -\langle w_{\phi}, u_0 \rangle - \beta A'(w_\phi, u_0; u^{n+1})
  - \beta\langle w_\phi, f(u_0)^{\perp}\rangle \\
  & \qquad\qquad
  + \beta\langle \phi, gD'\rangle
  -R_u(w_\phi;u^{n+1},D^{n+1})
  \quad \forall \phi \in V_2.
\end{align}
Hence, to this order, the Schur complement equation becomes
\begin{align}
  \nonumber
  \frac{1}{\gamma}\Big(-\langle w_{\phi}, \mathcal{L}_0D' \rangle 
  - \beta A'(w_\phi, \mathcal{L}_0D'; u^{n+1}) \quad & \\
  \nonumber
\qquad  \qquad   - \beta\langle w_\phi, f(\mathcal{L}_0D')^{\perp}\rangle
  + \beta\langle \phi, gD'\rangle\Big)
  &= \frac{1}{\gamma}\Big(R_u(w_\phi;u^{n+1},D^{n+1})
  -\langle w_{\phi}, r \rangle  \\
  & \qquad\qquad -\beta A'(w_\phi, r; u^{n+1})
  - \beta\langle w_\phi, fr^{\perp}\rangle\Big),
  \quad \forall \phi \in V_2,
\end{align}
$\mathcal{L}_0D'$ keeps the same or reduces the number of derivatives.
Further, the operator $\phi\to w_\phi$ is like a negative gradient.
This means that the operator with the most derivatives is the
mass operator (fourth term).

\section{Rotating compressible Euler equations}

We can write the Newton iteration for the implicit midpoint rule for
compatible FEM discretisation of the rotating compressible Euler equations
in the form
\begin{align}
  B(w,(u',\Pi',\theta'); \bar{U}) & = -R_u[w], \quad \forall w\in V_2, \\
  T(q,(u',\theta'); \bar{U}) & = -R_\theta[q], \quad \forall q\in V_\theta, \\
  C(\phi,(u',\rho'); \bar{U}) & = -R_\rho[\phi], \quad \forall \phi \in V_3, \\
  P(\mu,(\rho',\theta',\Pi'); \bar{U}) & = -R_\Pi[\mu], \quad
  \forall \mu \in V_3,
\end{align}
where $\bar{U}=(\bar{u},\bar{\theta},\bar{\rho},\bar{\Pi})$.

Now, to make the augmented formulation, we take a mapping $\tau:V_2\to
V_\theta$ that extracts the vertical part $w_v$, and transforms it into the
equivalent $V_\theta$ function (i.e. removes $J$ quantities and takes
the scalar part). Then, we augment the $u$ equation with all three
other equations, to get
\begin{align}
  \nonumber
  B(w,(u',\Pi',\theta'); \bar{U})
  +\gamma T(\tau w,(u',\theta'); \bar{U}) & \\
  \nonumber
  \qquad +\gamma C(\nabla \cdot w_v,(u',\rho'); \bar{U})
  +\gamma P(\nabla\cdot w_v,(\rho',\theta',\Pi'); \bar{U})
  & = -R_u[w]
  -\gamma R_\theta[\tau w] & \\
 & \qquad  -\gamma R_\rho[\nabla\cdot w_v]
  -\gamma R_\Pi[\nabla\cdot w_v], \quad \forall w\in V_2, \\
  T(q,(u',\theta'); \bar{U}) & = -R_\theta[q], \quad \forall q\in V_\theta, \\
  C(\phi,(u',\rho'); \bar{U}) & = -R_\rho[\phi], \quad \forall \phi \in V_3, \\
  P(\mu,(\rho',\theta',\Pi'); \bar{U}); \bar{U}) & = -R_\Pi[\mu], \quad
  \forall \mu \in V_3.
\end{align}
Then we consider a Schur complement preconditioner eliminating
$(u',\rho',\theta')$ to get a single equation for $\Pi'$, and look for
an approximate Schur complement. To analyse the Schur complement (and
suggest an approximation), we choose $w\in V_2^v$ (the vertical part
of $V_2$, so that $w=w_v$), and send $\gamma \to \infty$, to get
\begin{equation}
T(\tau w,(u',\theta'); \bar{U})
+C(\nabla \cdot w_v,(u',\rho'); \bar{U})
+P(\nabla\cdot w_v,(\rho',\theta',\Pi'); \bar{U})
 = 
- R_\theta[\tau w] - R_\rho[\nabla\cdot w_v]
- R_\Pi[\nabla\cdot w_v], \quad \forall w\in V_2^v.
\end{equation}
Choosing $w \in V_2^h$ (horizontal part) gives
\begin{equation}
  B(w, (u',\pi',\theta'); \bar{U}) = R_u[w], \quad \forall w\in
  V_2^h.
\end{equation}
Similar to the rotating shallow water equations, we define
$\mathcal{L}:V_2 \to V_1\times V_2 \times V_\theta$ with
$\mathcal{L}:\Pi \mapsto (\mathcal{L}_u\Pi,\mathcal{L}_\rho\Pi,
\mathcal{L}_\theta\Pi)$, according to
\begin{align}
  \nonumber
T(\tau w,(\mathcal{L}_u\Pi,\mathcal{L}_\theta\Pi; \bar{U})
+C(\nabla \cdot w_v,(\mathcal{L}_u\Pi,\mathcal{L}_\rho\Pi); \bar{U}) & \\
\qquad +P_1(\nabla\cdot w_v,(\mathcal{L}_\rho\Pi,\mathcal{L}_\theta\Pi); \bar{U})
& =
- P_2(\nabla\cdot w_v,\Pi'); \bar{U}), \quad \forall w\in V_2^v, \\ 
B_1(w, (\mathcal{L}_u\Pi,\mathcal{L}_\theta\Pi); \bar{U}) & =
-B_2(w, (u',\Pi'); \bar{U}), \quad \forall w\in
  V_2^h,
\end{align}
and $(r_u,r_\rho,r_\theta)\in V_1\times V_2 \times V_\theta$ according to
\begin{align}
  \nonumber
T(\tau w,(r_u,r_\theta; \bar{U})
+C(\nabla \cdot w_v,(r_u,\rho); \bar{U})
+P_1(\nabla\cdot w_v,(r_\rho,r_\theta); \bar{U})
 & = 
- R_\theta[\tau w] - R_\rho[\nabla\cdot w_v] \\
& \quad - R_\Pi[\nabla\cdot w_v], \quad \forall w\in V_2^v, \\ 
  B_1(w, (r_u,r_\theta); \bar{U}) & = R_u[w], \quad \forall w\in
  V_2^h,
\end{align}
where we have split up $P=P_1+P_2$ and $B=B_1+B_2$ into the part
containing $u'$, $\rho'$ and $\theta'$ and the part containing $\Pi'$.
Then, at leading order in $\gamma$, we have $(u',\rho'\theta')=
(\mathcal{L}_u\Pi+r_u,\mathcal{L}_\rho\Pi+r_\rho,\mathcal{L}_\theta\Pi+r_\theta)$.
This equation seems solvable (we don't need to solve it in practise, only in
theory) because the mass term in $T$ turns into a mass term for the
vertical part of $u'$.

Now we define $\phi_\gamma$ by finding $w\in V_1^v$ such that
$\tau w=\gamma$, and $\phi_\gamma = w$. Then we can conclude that
\begin{equation}
T(\gamma,(r_u,r_\theta; \bar{U})
+C(\phi_\gamma,(r_u,\rho); \bar{U})
+P_1(\phi_\gamma,(r_\rho,r_\theta); \bar{U})
  = 
- R_\theta[\gamma] - R_\rho[\phi_\gamma]
- R_\Pi[\phi_\gamma], \quad \forall \gamma\in V_\theta.
\end{equation}
Combining through the $\rho$ and $\Pi$ equations, we end up with
$0=0$ at leading order in $\gamma$, so we need to go the next order.

Writing $u'=u_0 + \gamma^{-1}u_1 + \mathcal{O}(\gamma^{-2})$, and
similar for $\theta'$ and $\rho'$, at $\mathcal{O}(\gamma^{-1})$ we
get
\begin{align}
  \nonumber
  B(w,(u'_0,\Pi'_0,\theta'_0); \bar{U})
  +T(\tau w,(u'_1,\theta'_1); \bar{U}) & \\
  \nonumber
  \qquad + C(\nabla \cdot w_v,(u'_1,\rho'_1); \bar{U})
  + P(\nabla\cdot w_v,(\rho'_1,\theta'_1,\Pi'_1); \bar{U})
  & = -R_u[w], \quad \forall w\in V_2, \\
  T(q,(u'_1,\theta'_1); \bar{U}) & = 0, \quad \forall q\in V_\theta, \\
  C(\phi,(u'_1,\rho'_1); \bar{U}) & = 0, \quad \forall \phi \in V_3, \\
  P(\mu,(\rho'_1,\theta'_1,\Pi'_1); \bar{U}); \bar{U}) & = 0, \quad
  \forall \mu \in V_3.
\end{align}
Going through the elimination steps again, we get
\begin{equation}
  -B(w_\phi, (\mathcal{L}_u\Pi',\mathcal{L}_\theta\Pi',\Pi'); \bar{U})
  = R_u[w_\phi],
  \quad \forall \phi \in V_3.
\end{equation}

\end{document}
