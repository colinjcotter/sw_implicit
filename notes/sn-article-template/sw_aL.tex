%%%%%%%%%%%%%%%%%%%%%%%%%%%%%%%%%%%%%%%%%%%%%%%%%%%%%%%%%%%%%%%%%%%%%
%%                                                                 %%
%% Please do not use \input{...} to include other tex files.       %%
%% Submit your LaTeX manuscript as one .tex document.              %%
%%                                                                 %%
%% All additional figures and files should be attached             %%
%% separately and not embedded in the \TeX\ document itself.       %%
%%                                                                 %%
%%%%%%%%%%%%%%%%%%%%%%%%%%%%%%%%%%%%%%%%%%%%%%%%%%%%%%%%%%%%%%%%%%%%%

%%\documentclass[referee,sn-basic]{sn-jnl}% referee option is meant for double line spacing

%%=======================================================%%
%% to print line numbers in the margin use lineno option %%
%%=======================================================%%

%%\documentclass[lineno,sn-basic]{sn-jnl}% Basic Springer Nature Reference Style/Chemistry Reference Style

%%======================================================%%
%% to compile with pdflatex/xelatex use pdflatex option %%
%%======================================================%%

%%\documentclass[pdflatex,sn-basic]{sn-jnl}% Basic Springer Nature Reference Style/Chemistry Reference Style

%%\documentclass[sn-basic]{sn-jnl}% Basic Springer Nature Reference Style/Chemistry Reference Style
\documentclass[sn-mathphys]{sn-jnl}% Math and Physical Sciences Reference Style
%%\documentclass[sn-aps]{sn-jnl}% American Physical Society (APS) Reference Style
%%\documentclass[sn-vancouver]{sn-jnl}% Vancouver Reference Style
%%\documentclass[sn-apa]{sn-jnl}% APA Reference Style
%%\documentclass[sn-chicago]{sn-jnl}% Chicago-based Humanities Reference Style
%%\documentclass[sn-standardnature]{sn-jnl}% Standard Nature Portfolio Reference Style
%%\documentclass[default]{sn-jnl}% Default
%%\documentclass[default,iicol]{sn-jnl}% Default with double column layout

%%%% Standard Packages
%%<additional latex packages if required can be included here>
%%%%

%%%%%=============================================================================%%%%
%%%%  Remarks: This template is provided to aid authors with the preparation
%%%%  of original research articles intended for submission to journals published 
%%%%  by Springer Nature. The guidance has been prepared in partnership with 
%%%%  production teams to conform to Springer Nature technical requirements. 
%%%%  Editorial and presentation requirements differ among journal portfolios and 
%%%%  research disciplines. You may find sections in this template are irrelevant 
%%%%  to your work and are empowered to omit any such section if allowed by the 
%%%%  journal you intend to submit to. The submission guidelines and policies 
%%%%  of the journal take precedence. A detailed User Manual is available in the 
%%%%  template package for technical guidance.
%%%%%=============================================================================%%%%

\jyear{2021}%

%% as per the requirement new theorem styles can be included as shown below
\theoremstyle{thmstyleone}%
\newtheorem{theorem}{Theorem}%  meant for continuous numbers
%%\newtheorem{theorem}{Theorem}[section]% meant for sectionwise numbers
%% optional argument [theorem] produces theorem numbering sequence instead of independent numbers for Proposition
\newtheorem{proposition}[theorem]{Proposition}% 
%%\newtheorem{proposition}{Proposition}% to get separate numbers for theorem and proposition etc.

\theoremstyle{thmstyletwo}%
\newtheorem{example}{Example}%
\newtheorem{remark}{Remark}%

\theoremstyle{thmstylethree}%
\newtheorem{definition}{Definition}%

\newcommand{\pp}[2]{\frac{\partial #1}{\partial #2}} 
\DeclareMathOperator{\diff}{d}
\newcommand{\jump}[1]{[\![ #1]\!]} 
\raggedbottom
%%\unnumbered% uncomment this for unnumbered level heads

\begin{document}

\title[Augmented Lagrangian shallow water]{Augmented Lagrangian preconditioning implicit time solvers for the rotating shallow water equations on the sphere}

\author*[1]{\fnm{First} \sur{Author}}\email{iauthor@gmail.com}

%\author[2,3]{\fnm{Second} \sur{Author}}\email{iiauthor@gmail.com}
%\equalcont{These authors contributed equally to this work.}

%\author[1,2]{\fnm{Third} \sur{Author}}\email{iiiauthor@gmail.com}
%\equalcont{These authors contributed equally to this work.}

\affil*[1]{\orgdiv{Department of Mathematics}, \orgname{Imperial College London}, \orgaddress{\street{South Kensington Campus}, \city{London}, \postcode{SW7 2AZ}, \country{United Kingdom of Great Britain and Northern Ireland}}}

%\affil[2]{\orgdiv{Department}, \orgname{Organization}, \orgaddress{\street{Street}, \city{City}, \postcode{10587}, \state{State}, \country{Country}}}

%\affil[3]{\orgdiv{Department}, \orgname{Organization}, \orgaddress{\street{Street}, \city{City}, \postcode{610101}, \state{State}, \country{Country}}}

%%==================================%%
%% sample for unstructured abstract %%
%%==================================%%

\abstract{ We describe augmented Lagrangian for fully implicit time
  discretisations of the rotating shallow water equations on the
  sphere, where the spatial discretisation is constructed using
  compatible finite element methods. The solver allows for large
  timesteps, in fact the convergence is quicker with large $\Delta t$,
  and the convergence rate is mesh independent. We demonstrate these
  claims using numerical experiments.}

%%================================%%
%% Sample for structured abstract %%
%%================================%%


\keywords{keyword1, Keyword2, Keyword3, Keyword4}

%%\pacs[JEL Classification]{D8, H51}

%%\pacs[MSC Classification]{35A01, 65L10, 65L12, 65L20, 65L70}

\maketitle

\section{Introduction}\label{sec:intro}

\section{Formulation}\label{sec:formul}

In this article we restrict the focus to compatible finite elements,
i.e. finite elements derived from a discrete de Rham complex as
detailed in \citet{arnold2006finite,arnold2018finite}, and discussed
for geophysical fluid dynamics in
\citet{cotter2012mixed,gibson2019compatible}.

We consider the shallow water equations on the rotating sphere (centre
the origin, with radius $R$),
\begin{align}
  u_t + (u\cdot \nabla)u + fu^{\perp} + g\nabla (D+b) & = 0, \\
  D_t + \nabla(uD) & = 0, \Omega
\end{align}
where $u$ is the velocity (constrained to be tangential to the
sphere), $f=2\Omega z/R$ is the Coriolis parameter (for a sphere with
rotation rate $\Omega$), $u^{\perp}=k\times u$ where $k$ is the unit
vector tangential to the surface of the sphere, $g$ is the
acceleration due to gravity, $D$ is the depth of the fluid, and $b$ is
the height of the bottom boundary.

There are a few different ways to treat the advective nonlinearity
numerically, and for example, we will use the spatial discretisation
of \citet{gibson2019compatible}, namely
\begin{align}
  \int_{\Omega}{w}\cdot\pp{{u}}{t}\diff x
  - \int_\Omega {u}\cdot\nabla^\perp ({u}^\perp\cdot{w})\diff x
  + \int_\Gamma\left[{n}^\perp({u}^\perp\cdot{w})\right]\cdot\widetilde{{u}}\diff S\nonumber\\
  + \int_\Omega{w}\cdot (f{u}^\perp) \diff x
  - \int_{\Omega}\nabla\cdot{w} \left(g(h + \eta_b)
  + \frac{1}{2}|{u}|^2 \right)\diff x &=0,\label{eq:weakVISWE-A}\\
  \int_{\Omega}\phi\pp{h}{t}\diff x - \int_{\Omega}h({u}\cdot\nabla)\phi\diff x
  + \int_{\Gamma} \jump{{u}\phi}\widetilde{h}\diff x
  &= 0.\label{eq:weakVISWE-B}
\end{align}

\section{Numerical experiments}\label{sec:numerics}

\backmatter

\bmhead{Supplementary information}

If your article has accompanying supplementary file/s please state so here. 

Please refer to Journal-level guidance for any specific requirements.

\bmhead{Acknowledgments}

Acknowledgments are not compulsory. Where included they should be brief. Grant or contribution numbers may be acknowledged.

Please refer to Journal-level guidance for any specific requirements.

\section*{Declarations}

There are no competing interests.

%% Some journals require declarations to be submitted in a standardised format. Please check the Instructions for Authors of the journal to which you are submitting to see if you need to complete this section. If yes, your manuscript must contain the following sections under the heading `Declarations':

%% \begin{itemize}
%% \item Funding
%% \item Conflict of interest/Competing interests (check journal-specific guidelines for which heading to use)
%% \item Ethics approval 
%% \item Consent to participate
%% \item Consent for publication
%% \item Availability of data and materials
%% \item Code availability 
%% \item Authors' contributions
%% \end{itemize}

%% \noindent
%% If any of the sections are not relevant to your manuscript, please include the heading and write `Not applicable' for that section. 

%%===================================================%%
%% For presentation purpose, we have included        %%
%% \bigskip command. please ignore this.             %%

\begin{appendices}

\section{Section title of first appendix}\label{secA1}

An appendix contains supplementary information that is not an essential part of the text itself but which may be helpful in providing a more comprehensive understanding of the research problem or it is information that is too cumbersome to be included in the body of the paper.

%%=============================================%%
%% For submissions to Nature Portfolio Journals %%
%% please use the heading ``Extended Data''.   %%
%%=============================================%%

%%=============================================================%%
%% Sample for another appendix section			       %%
%%=============================================================%%

%% \section{Example of another appendix section}\label{secA2}%
%% Appendices may be used for helpful, supporting or essential material that would otherwise 
%% clutter, break up or be distracting to the text. Appendices can consist of sections, figures, 
%% tables and equations etc.

\end{appendices}

%%===========================================================================================%%
%% If you are submitting to one of the Nature Portfolio journals, using the eJP submission   %%
%% system, please include the references within the manuscript file itself. You may do this  %%
%% by copying the reference list from your .bbl file, paste it into the main manuscript .tex %%
%% file, and delete the associated \verb+\bibliography+ commands.                            %%
%%===========================================================================================%%

\bibliography{sn-bibliography}% common bib file
%% if required, the content of .bbl file can be included here once bbl is generated
%%\input sn-article.bbl

%% Default %%
%%\input sn-sample-bib.tex%

\end{document}
